\documentclass[a4paper]{article}
\usepackage[brazilian]{babel}
\usepackage{subfig}
\usepackage{booktabs}
\usepackage{graphicx}
%-----------------------------------------------------------------
\title{MS438 - Programa\c{c}\~ao Linear.}
\author{122830 Alcides Goldoni Junior\\
  \small Modelagem matem\'{a}tica \\
  %\small Unicamp 
}%Fechando Title
\begin{document}
\maketitle
%Seções
%-----------------------------------------------------------------
\section{Modelagem - Exemplos}
\subsection{O problema: F\'abrica de m\'oveis}
Uma f\'abrica tem em estoque 250m de t\'abuas, 600m de prancha e 500m de pain\'eis conglomerados. A f\'abrica oferece uma linha de m\'oveis composta de escrivaninha, mesa de reuni\~ao, arm\'ario e prateleiras. Cada tipo de m\'ovel é constru\'ido com uma certa quantidade de mat\'eria-prima, que \'e dado na tabela. O valor de venda de cada unidade de cada produto tamb\'em é dado na tabela.
\\
Escreva um modelo que maximize a receita com a venda dos m\'oveis.
\\
\begin{table}[]
\centering
\caption{Materiais da empresa}
\label{my-label}
\begin{tabular}{c|c|c|c|c|c}
\cline{2-5}
& \multicolumn{4}{c|}{Quantidade de material (m)} & \\ 
\cline{2-6} 
& Escriv.    & Mesa    & Arm\'ario   & Prateleira   & \multicolumn{1}{c|}{Dispon\'ivel} \\ 
\hline
\multicolumn{1}{|c|}{T\'abua}   & 1          & 1       & 1         & 4            & \multicolumn{1}{c|}{250}        \\ \hline
\multicolumn{1}{|c|}{Prancha} & 0          & 1       & 1         & 2            & \multicolumn{1}{c|}{600}        \\ \hline
\multicolumn{1}{|c|}{Painel}  & 3          & 2       & 4         & 0            & \multicolumn{1}{c|}{500}        \\ \hline
\multicolumn{1}{|c|}{Valor}   & 100        & 80      & 120       & 20           &                                 \\ \cline{1-5}
\end{tabular}
\end{table}
\\
\subsection{A solu\c{c}\~ao}
\begin{itemize}
\item Defini\c{c}\~oes
Vamos definir que $x_i$ \'e a quantidade de moveis produzida;
\\
$x_1$ = Escrivaninhas
\\
$x_2$ = Mesas 
\\
$x_3$ = Arm\'arios
\\
$x_4$ = Prateleiras
\\
\item Fun\c{c}\~ao Objetivo
\\
Nosso problema quer maximizar os lucros da empresa, portanto:
\\
\begin{equation}
max f(x_1,x_2,x_3,x_4) = 100x_1 + 80x_2 +120 x_3 + 20 x_4
\end{equation}
\item Restri\c{c}\~oes
\\
$f(x_1,x_2,x_3,x_3) = 
\left \{
\begin{array}{ll}
x_1 + x_2 + x_3 +4x_4 \leq 250 \\ 
0x_1 + x_2 + x_3 +1x_4  \leq 600\\ 
3x_1 + 2x_2 + 4x_3 +0x_4  \leq 500\\ 
x_1,x_2,x_3, x_4 \geq 0
\end{array}
\end{itemize}
\end{document}
